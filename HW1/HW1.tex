\documentclass[12pt]{article}
\usepackage[utf8]{inputenc}
\usepackage[english]{babel}

\usepackage{geometry}
\usepackage[document]{ragged2e}
\setlength{\parindent}{0.5in}
\setlength{\RaggedRightParindent}{\parindent}
\usepackage{indentfirst}

\usepackage{setspace}
\usepackage{titling}
\usepackage{fancyhdr}

\usepackage{lmodern}

\usepackage{graphicx}
\graphicspath{{./}}

\usepackage{titlesec}
\titleformat{\section}
	{\normalfont\normalsize\sc}{}{0pt}{\\}{}
	% format, label, space between number and text, before code, after code
\titlespacing{\section}
	{0pt}{0pt}{0pt}
	% left, top, bottom

% \pagestyle{fancy}
% \fancyhf{}
% \rhead{Atchley \thepage}
% \renewcommand{\headrulewidth}{0pt}

\newcommand{\problem}[2]{\bigskip \noindent \textbf{Homework problem 1.#1: #2}\par}
\newcommand{\reffig}[1]{Figure \ref{#1}}

\doublespacing
\geometry{letterpaper, margin=1in}
\setlength{\parindent}{0.5in}

\title{HW \#1: Model Building Blocks}
\author{Hayden Atchley}
\date{12 September 2021}

%%%%%%%%%%%%%%%%%%%%%%%%%%%%%%%%%%%%%%%%%%%%%%%%%

\begin{document}

\noindent\theauthor\\
\noindent Dr. Gregory Macfarlane\\
\noindent CE 565\\
\noindent\thedate\\
\noindent\centering\thetitle\\
\RaggedRight

\problem{1}{Socioeconomic data visualization}
	Using the TAZ layer and socioeconomic data given as inputs to the model, I created a set of maps showing various socioeconomic variables by zone.
	These variables are: Total households (\reffig{fig:hh}), Household density (\reffig{fig:hhdens}), Total jobs (\reffig{fig:jobs}), Job density (\reffig{fig:jobsdens}), and Employment by type (Figures \ref{fig:jobtype}, \ref{fig:jobtypezoom}).
	The location of the central business district (CBD) is clear from the map of job density, and is shown in \reffig{fig:cbd}.
	Additionally, there appears to be another employment center to the west of the CBD, and is shown as well (\reffig{fig:obd}).
	I also included a satellite image of the area from Google Maps (\reffig{fig:sat}) for reference.

	\begin{figure}[h]
		\centering
		\includegraphics[height=0.4\textheight]{1-1_pictures/HH}
		\caption{Total households shown by traffic analysis zone (TAZ)}
		\label{fig:hh}
	\end{figure}

	\begin{figure}[p]
		\centering
		\includegraphics[height=0.4\textheight]{1-1_pictures/HHDens}
		\caption{Household density shown by TAZ, households per acre}
		\label{fig:hhdens}
	\end{figure}

	\begin{figure}[p]
		\centering
		\includegraphics[height=0.4\textheight]{1-1_pictures/Jobs}
		\caption{Total jobs shown by TAZ}
		\label{fig:jobs}
	\end{figure}

	\begin{figure}[p]
		\centering
		\includegraphics[height=0.4\textheight]{1-1_pictures/JobDens}
		\caption{Job density shown by TAZ, jobs per acre}
		\label{fig:jobsdens}
	\end{figure}

	\begin{figure}[p]
		\centering
		\includegraphics[height=0.4\textheight]{1-1_pictures/JobTypes}
		\caption{Job type distribution shown by TAZ, as percentage of total jobs}
		\label{fig:jobtype}
	\end{figure}

	\begin{figure}[p]
		\centering
		\includegraphics[height=0.4\textheight]{1-1_pictures/JobTypesZoomed}
		\caption{Job type distribution shown by TAZ, as percentage of total jobs (zoomed in)}
		\label{fig:jobtypezoom}
	\end{figure}

	\begin{figure}[p]
		\centering
		\includegraphics[height=0.4\textheight]{1-1_pictures/CBD}
		\caption{Location of central business district with TAZ numbers}
		\label{fig:cbd}
	\end{figure}

	\begin{figure}[p]
		\centering
		\includegraphics[height=0.4\textheight]{1-1_pictures/OtherBD}
		\caption{Location of other employment center district with TAZ numbers}
		\label{fig:obd}
	\end{figure}

	\begin{figure}[p]
		\centering
		\includegraphics[height=0.4\textheight]{1-1_pictures/satellite}
		\caption{Satellite imagery of study area, courtesy of Google Maps}
		\label{fig:sat}
	\end{figure}

\pagebreak

\problem{2}{Link information visualization}
	Using the output network obtained from the model, I created maps based on various parameters of each link: Link type (\reffig{fig:linktype}), Link free flow speed (\reffig{fig:linkffs}), and Link hourly capacity (\reffig{fig:linkcap}).
	It is clear from these maps that a major freeway runs along the north edge of the study area, and a principal arterial runs north-south basically through the center of the city.
	Again a satellite image is given (\reffig{fig:sat}) for reference.
	
	\begin{figure}[h]
		\centering
		\includegraphics[height=0.4\textheight]{1-2_pictures/LinkType}
		\caption{Highway links by type}
		\label{fig:linktype}
	\end{figure}

	\begin{figure}[p]
		\centering
		\includegraphics[height=0.4\textheight]{1-2_pictures/LinkFFS}
		\caption{Highway links by free flow speed}
		\label{fig:linkffs}
	\end{figure}

	\begin{figure}[p]
		\centering
		\includegraphics[height=0.4\textheight]{1-2_pictures/LinkCAP}
		\caption{Highway links by capacity}
		\label{fig:linkcap}
	\end{figure}

\pagebreak

\problem{3}{Shortest paths}
	In CUBE I found the shortest distance between Zone 195 and Zone 109, using both a free flow speed calculation (\reffig{fig:shortffs}) and a raw distance calculation (\reffig{fig:shortdist}).
	The two paths differ considerably from each other, as the path using the free flow speed calculation uses a major road that has a much higher speed than the local roads used in the raw distance calculation.
	I have included a path calculation from Google maps between these same zones around 2AM (\reffig{fig:shortmap}), and it matches rather closely the path from the free flow speed calculation.
	
	\begin{figure}[h]
		\centering
		\includegraphics[height=0.4\textheight]{1-3_pictures/Short_FFS}
		\caption{Shortest path between TAZ 195 and 109 by free flow speed}
		\label{fig:shortffs}
	\end{figure}

	\begin{figure}[p]
		\centering
		\includegraphics[height=0.4\textheight]{1-3_pictures/195-109_dist}
		\caption{Shortest path between TAZ 195 and 109 by raw distance}
		\label{fig:shortdist}
	\end{figure}

	\begin{figure}[p]
		\centering
		\includegraphics[height=0.4\textheight]{1-3_pictures/shortmap}
		\caption{Shortest path between TAZ 195 and 109 according to Google Maps}
		\label{fig:shortmap}
	\end{figure}

\pagebreak

\problem{4}{Highway Assessment Report}
	The highway assessment report (an output of the model) contains information about vehicle hours and miles traveled by facility type.
	The report is tabulated below in Table \ref{tab:har}.
	The table has been editorialized to clarify the original abbreviations used.
	
	\begin{table}[h]
		\centering
		\caption{Highway assessment report}
		\label{tab:har}
		\bigskip
	\begin{tabular}{l c c}
		Facility Type & Vehicle Miles Traveled & Vehicle Hours Traveled \\
		\hline
		1 (Principal Freeway) & 1799198 & 26710 \\
		2 (Minor Freeway) & 134847 & 2250 \\
		3 (Principal Arterial) & 722354 & 16404 \\
		4 (Major Arterial) & 113722 & 3571 \\
		5 (Minor Arterial) & 819899 & 21891 \\
		6 (Major Collector) & 243297 & 7004 \\
		7 (Minor Collector) & 92553 & 2522 \\
		8 (Local) & 48850 & 1771 \\
		9 (High-speed Ramp) & 24942 & 443 \\
		10 (Low-speed Ramp) & 52382 & 1475 \\
		11 (Centroid Connector) & 0 & 0 \\
		12 (External Station Connector) & 0 & 0 \\
		\hline
		Total & 4052043 & 84042
	\end{tabular}
	\end{table}
	
	From this table we can see that \(1 799 198 + 134 847 = 1 934 045\) miles were traveled on freeways, which is \(\frac{1 934 045}{4 052 043} = 47.7\%\) of the total miles traveled.
	
\pagebreak

\problem{5}{PM Level of Service}
	Using the model's output volume to capacity ratios, I created a map (\reffig{fig:los}) based on the level of service (LOS) values given in Table \ref{tab:los} below:
	
	\begin{table}[h]
		\centering
		\caption{Level of service values}
		\label{tab:los}
		\bigskip
		\begin{tabular}{c c}
			LOS & V/C \\
			\hline
			A & \(< 0.35\) \\
			B & 0.35\textendash0.54 \\
			C & 0.55\textendash0.77 \\
			D & 0.78\textendash0.93 \\
			E & 0.94\textendash0.99 \\
			F & \(\geq 1.00\)
		\end{tabular}
	\end{table}

	\begin{figure}[h]
		\centering
		\includegraphics[height=0.4\textheight]{1-5_pictures/LinkLOS}
		\caption{Level of service of each highway link}
		\label{fig:los}
	\end{figure}

	From this map it would appear that the roads in the study area are all very over-engineered.
	However, I believe the reason for this is that we have not yet calibrated the model correctly, so the actual LOS may be very different once the model is calibrated.


\end{document}